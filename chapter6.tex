%chapter 3
\chapter{Conclusion and Future Work}
\label{future_work}
This chapter discusses what has been presented here and possible future extensions of this dissertation.

\section{Summary}
This dissertation analyzes the impact of boulders on asteroid dynamics, puts them in the context of observability from shape-from-silhouette shape modeling, and discusses the possible dynamical outcomes of boulder motion. Using images to build a shape model as well as to inform boulder statistics, we investigated details of the shape and the impact of the resolution of both the base shape model and the boulder diameters. We aimed to provide detail to the concept that the YORP effect is uncertain and stochastically applied in long simulations of asteroid family dynamics. To do so, we sourced in-situ observations of asteroid boulders to inform a simulation model. We assumed a wedge model for inducement of orientation asymmetry, and imposed size, location, and orientation distributions on our full population. We simulated 500 models of 5000 boulders on two separate bodies of similar radii to investigate the impact of convex versus irregular shaped bodies on the boulder-induced YORP effect. These simulations calculated the YORP coefficients $C_0, C_1,$ and $D_1$ in order to find the rate of angular velocity and rate of solar inclination change for each facet of the shape and boulders. This expanded the computational effort of calculating YORP by 8 times due to the consideration of 8x5000 more facets from boulders versus the $\sim$5000 facet base shape models. 
Further analysis was done to induce biasing in the locations and orientations of boulders to find the sensitivity of YORP spin acceleration to these factors. We've shown how boulder motion in latitude can accelerate the YORP timescale or induce tumbling modes. The statistics presented in this work will inform ground-based YORP estimates for bodies in the rubble-pile size regime.

\section{Applications of Boulder YORP Modeling}
One application of this work is to compare ground-based YORP measurements of Bennu with the highly detailed measurements made by OSIRIS-REx and make up the difference in measurements with randomized boulder models. The same can be done for Itokawa which has ground-based and space-based shape models as well as boulder statistics and boulder-induced YORP calculations shown here. Through this verification, it can be shown what extreme boulder cases are possible when trying to solve for the difference between ground and space based YORP measurements. The solution of boulder size and locations from YORP measurements will never be fully unique, but additionally applying knowledge of the geology of local materials and their breakup potential, it can be narrowed down that from a maximum boulder size of tens of meters, a certain percentage of them need to be in specific latitude ranges to induce observed YORP accelerations. Each new observation of an asteroid and measurement of its YORP acceleration opens up the possibilities for applying this statistical analysis.

\section{Further Investigations}
Future extensions of this work can involve inducing a time-varying thermal inertia. The assumption made in our equations of motion is that we can use a 1/8th period equivalent thermal lag for the obliquity varying equations. This approach would require re-deriving the equations of motion from Scheeres will retaining the consideration of a time-varying, or material-varying thermal inertia value.
Other extensions of this work could include the integration of optical and thermal-based limb observations. Sensor fusion is often applied to enhance spacecraft navigation data products such as the trajectory, and in the case of shape modeling, one could observe the unlit side of the body which is still differentiable from space in the infrared wavelength.
Another future work possibility includes the integration of boulder motion in the dynamical sense, of propagating the asteroid spin and orbit dynamics through time and initiating a surface redistribution when facets reach their cohesive or friction limit. This is highly applicable to cases such as Apophis, which is an asteroid on course to come near the Earth in 2029. Planet conjunctions can induce large gravitational and centripetal forces through the hyperbolic flyby that could easily redistribute boulders on the surface. While this isn't expected for Apophis specifically, it is an interesting problem for other more dynamic surfaces.

