\documentclass[defaultstyle,11pt]{thesis}

\usepackage{amssymb}		% to get all AMS symbols
\usepackage{graphicx}		% to insert figures
%\usepackage{hyperref}		% PDF hyperreferences??
\usepackage[backend=bibtex,style=authoryear,maxbibnames=3,natbib=true]{biblatex}
%\bibliographystyle{authoryear}
\addbibresource{dissertation}
\addbibresource{chapter2bib}
%\setcitestyle{open={(},author,year,close={)}} 
%\bibliographystyle{authoryear}
%\usepackage[backend=biber,style=authoryear,minnames=1,maxcitenames=4]{biblatex}
%\setcitestyle{authoryear,open={(},close={)}} 
\usepackage{caption}
\usepackage{subcaption}
\usepackage{wrapfig}
\usepackage{amsmath}
\usepackage{mathtools}
\usepackage{longtable,tabularx}
\usepackage{float}
\usepackage{makecell}
\usepackage{siunitx}
\usepackage[T1]{fontenc}

%%%%%%%%%%%%   All the preamble material:   %%%%%%%%%%%%

\title{The Influence of Asteroid Surface Features on the YORP Effect and Dynamical Evolution}

\author{Dahlia A.}{Baker}

\otherdegrees{B.A., Coe College, 2018 \\
	      M.S., University of Colorado Boulder, 2021}

\degree{Doctor of Philosophy}		%  #1 {long descr.}
	{Ph.D., Aerospace Engineering Sciences}		%  #2 {short descr.}

\dept{Ann and H.J. Smead Department}			%  #1 {designation}
	{of Aerospace Engineering Sciences}		%  #2 {name}

\advisor{}				%  #1 {title}
	{Dr. Jay W. McMahon}			%  #2 {name}

\reader{Dr. Daniel J. Scheeres}		%  2nd person to sign thesis
\readerThree{Dr. Paul Sanchez}		%  3rd person to sign thesis
\readerFour{Dr. Paul Hayne}
\readerFive{Dr. William Bottke}

\abstract{  % because it is very short

	Small rubble-pile asteroids have rocky, rough surfaces and a wide variety of possible shapes that make them unique targets for mapping and estimating their dynamical evolution. The surface undergoes thermal interactions that induce a small additional torque over time. These small torques are the YORP effect, or re-radiated thermal energy from the surface that adds small torques to the spin pole of the body over time and is dependent on the surface material's thermal inertia and the asteroid's spin velocity and spin pole orientation. Much like solar radiation pressure, this interaction with the sun induces perturbations on the spin that can lead bodies secularly towards chaotic or destructive spin states. In order to understand their evolution towards these states, we investigate the relationship between the shape and dynamics. Modeling their shapes accurately requires knowledge of the surface through imagery, and then this contributes to better models of their dynamics since the shape and dynamics are coupled through the thermal interactions of the YORP effect. The limb can show a unique roughness pattern that can then be translated as a feature in three-dimensional space by leveraging what is known about the asteroid spin and orbit state as well as the camera or spacecraft attitude and trajectory. The work presented in this dissertation shows a method for dissecting a limb measurement into many points, associating them from the image frame to the three-dimensional body frame, and performing outlier rejection and shape reconstruction from the many observed points constraining the surface. In this effort, we show our shape modeling resolution and error to evaluate what the upper limit of boulder resolution could be with our methods. This upper limit informs what the largest observable boulder is, and then that is used to inform how accurate of a YORP prediction we are able to make by including boulders down to this size limit. This informs our overall uncertainty in YORP predictions when a shape model is accessible but the boulder population is not known accurately.
	We model the impact of boulders on the YORP torque by applying the boulder statistics from mapped asteroid surfaces, adding statistically sampled boulder models to Bennu and Itokawa shape models, and randomizing many of these models to find the likelihood of large boulder-induced changes in the spin velocity and spin pole orientation dynamics. 
	We have found that we can narrow down that YORP-induced torque from boulders is only influential from boulders 1m in diameter and above (equivalently stated as boulders sized at 1/500th of the asteroid diameter). This size regime makes up just $10\%$ of the boulder population based on Bennu, Ryugu, and Itokawa. This reduces the number of features that one is required to model in order to obtain more accurate YORP estimates for a body. However, it also supports the conclusion that single boulders do not dominate the YORP prediction and that multiple of the largest and best placed boulders need to be known about and modeled in order to understand the net effect of YORP from the entire surface population. This also takes into consideration that the largest boulders on the surface are typically stationary, contributing the same influence to YORP over time, while smaller influential boulder motion can cause the walk seen in stochastic YORP.
	}


















\dedication[Dedication]{	% NEVER use \OnePageChapter here.
	I dedicate this manuscript to my Mom. Thank you for pushing me to strive for great things.
	}

\acknowledgements{	\OnePageChapter	% *MUST* BE ONLY ONE PAGE!
	I'd like to acknowledge my tireless advisor, Jay McMahon, for bringing me into the ORCCA lab and letting me do whatever I wanted with my research while supporting me at every step along the way. I have many exciting opportunities ahead of me thanks to you. I am also very grateful to my committee members, Dr. Scheeres, Dr. Hayne, Dr. S\'{a}nchez, and Dr. Bottke for their guidance and expertise which shapes this work. 
	I thank my colleagues in the ORCCA Lab and CCAR whose advice, collaboration, and kindness made this all possible. Lyss, you deserve the so much recognition for your compassion and brilliance. Also, my close friend Liane, for giving me so many recipes and endless support and mentorship. Without your friendships, I would not have made it through.
	I would also like to recognize my professors from the Coe College physics department, who pushed me to continue with my education and became my family in Iowa. Thank you Ugur, Mario, Doc, and Firdevs for your dedication and guidance. Your commitment to student success and excellence in research let me chase my dreams. 
	Another thank you goes to my partner, Nicholas, for your patience and support as I approach graduation and take on a job across the country. Lastly, I want to thank my all of my family and friends who are reading this, who know how difficult this has been and have been there for me in all of the hard times. 
}	

% \IRBprotocol{E927F29.001X}	% optional!

%\ToCisShort	% use this only for 1-page Table of Contents

%\LoFisShort	% use this only for 1-page Table of Figures
% \emptyLoF	% use this if there is no List of Figures

\LoTisShort	% use this only for 1-page Table of Tables
% \emptyLoT	% use this if there is no List of Tables

%%%%%%%%%%%%%%%%%%%%%%%%%%%%%%%%%%%%%%%%%%%%%%%%%%%%%%%%%%%%%%%%%
%%%%%%%%%%%%%%%       BEGIN DOCUMENT...         %%%%%%%%%%%%%%%%%
%%%%%%%%%%%%%%%%%%%%%%%%%%%%%%%%%%%%%%%%%%%%%%%%%%%%%%%%%%%%%%%%%

\begin{document}

\input macros.tex
\input chapter1.tex
\input chapter3.tex
\input chapter4.tex
\input chapter2.tex
%\input chapter5.tex
\input chapter6.tex

%%%%%%%%%   then the Bibliography, if any   %%%%%%%%%
%\bibliographystyle{alpha}	% or "siam", or "alpha", etc.
%\nocite{*}		% list all refs in database, cited or not
%\bibliography{dissertation,chapter2bib}		% Bib database in "refs.bib"
\printbibliography
%%%%%%%%%   then the Appendices, if any   %%%%%%%%%
%\appendix
%\chapter{Extra Things}	% *NOT* \OnePageChapter

\paragraph{About appendices:}
	Each appendix follow the same page-numbering rules
	as a regular chapter; the first page of a
	(multi-page) appendix is not numbered.
	By the way, the following are supposedly
	authentic answers to English GCSE exams!


\begin{enumerate}

\item
The Greeks were a highly sculptured people, and without
them we wouldnt have history. The Greeks also had myths.
A myth is a female moth.

\item
Actually, Homer was not written by Homer but by another
man of that name.

\item
Socrates was a famous Greek teacher who went around
giving people advice. They killed him. Socrates died from an
overdose of wedlock. After his death, his career suffered a
dramatic decline.

\item
Julius Caesar extinguished himself on the battlefields
of Gaul. The Ides of March murdered him because they thought
he was going to be made king. Dying, he gasped out: Tee hee,
Brutus.

\item
Nero was a cruel tyranny who would torture his subjects
by playing the fiddle to them.

\item
In midevil times most people were alliterate. The
greatest writer of the futile ages was Chaucer, who
wrote many poems and verses and also wrote literature.

\item
Another story was William Tell, who shot an arrow
through an apple while standing on his sons head.

\item
Writing at the same time as Shakespeare was Miguel
Cervantes. He wrote Donkey Hote. The next great author
was John Milton. Milton wrote Paradise Lost. Then his
wife died and he wrote Paradise Regained.

\item
During the Renaissance America began. Christopher
Columbus was a great navigator who discovered America while
cursing about the Atlantic. His ships were called the Nina,
the Pinta, and the Santa Fe.

\item
Gravity was invented by Issac Walton. It is chiefly
noticeable in the autumn when the apples are falling
off the trees.

\item
Johann Bach wrote a great many musical compositions and
had a large number of children. In between he practiced on
an old spinster which he kept up in his attic. Bach died
from 1750 to the present. Bach was the most famous composer
in the world and so was Handel. Handel was half German
half Italian and half English. He was very large.

\item
Soon the Constitution of the United States was adopted
to secure domestic hostility. Under the constitution the
people enjoyed the right to keep bare arms.

\item
The sun never set on the British Empire because the
British Empire is In the East and the sun sets in the West.

\item
Louis Pasteur discovered a cure for rabbis. Charles
Darwin was a naturalist who wrote the Organ of the Species.
Madman Curie discovered radio. And Karl Marx became one of
the Marx brothers.

\end{enumerate}


\end{document}

